\chapter{Conclusion}
\label{ch:conclusion}
As mentioned in the introduction, the goal of this thesis is twofold:
\begin{enumerate}
    \item To develop an algorithmic model of a battlefield with opposing agents and a terrain that can both prohibit movement and visibility.
    \item To assess the feasibility of using state-of-the-art RL algorithms on this model to develop strategies.
\end{enumerate}
The first goal has been met: an algorithmic model has been designed and implemented that tries to take into account the most important elements of real-life combat situations. Furthermore, the implementation has been done in such a way that the model can be easily extended. These extensions may include things like additional state variables (e.g. fuel consumption), the addition of new restrictions on the behavior of agents, the addition of extra actions that an agent may take, creating obstacles that inhibit movement but not visibility or vice versa, different kinds of agents, \ldots\\
As for the second goal, a number of reinforcement algorithms have been implemented. These include algorithms from the policy gradient family that directly try to push the agents policy in the right direction, like REINFORCE or Actor-Critic algorithms, and algorithms from the Q-learning family where the goal is to make an accurate estimation of the Q-value and derive a policy from that. Examples of the latter family are Independent Q-learning and QMix.\\
The results of these algorithms is mixed. Purely based on performance, the simplest of these algorithms, REINFORCE, performs best. This might be because it's the best algorithm, because it is less sensitive to hyperparameters and the optimal set of hyperparameters for the other algorithms has not been found, or due to implementation issues with the other algorithms.\\
QMix is true multi-agent algorithm, developed to stimulate coordination between agents and by doing so to develop coordinated strategies. However, the limited experimentation performed with QMix on the environment model has not definitely shown this coordination. Furthermore, QMix shows to have convergence problems when the board size becomes larger.\\
Care was taken during implementation to keep both the neural network and the observations for the different agents as generic as possible. This made it possible to easily transfer networks between agents, even of different teams. This in turn allowed the use of model transfer to learn to play against teams of increasing strength. This procedure works well and produces good result. However, more evaluation of the developed strategies is needed.\\
Personally, I would like to say that implementing these algorithms from scratch is not always an easy or straightforward feat. Some papers don't always make explicit which technical assumptions have been made, thus during implementation certain well considered choices had to be made in the hope that they align with the intention of the authors. Another difficulty is the large number of hyperparameters that each algorithm has. As mentioned previously, choosing a good combination of hyperparameters can be hard, an certain algorithms are very sensitive to a good choice.\\
This work provided several contributions to the Belgian Defence:
\begin{itemize}
    \item Participation in the IRIS project mentioned in the introduction.
    \item Development of a certain expertise in a domain that becomes more and more important.
    \item The start of the development of a tool that has the potential to contribute significantly to the decision making process.
\end{itemize}
